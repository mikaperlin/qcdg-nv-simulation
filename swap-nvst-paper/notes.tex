\documentclass[twocolumn]{revtex4-1}

% \usepackage{showframe}

\usepackage{hyperref} % for linking references

%%% math, symbols, etc.
\usepackage{physics,braket,bm,amssymb}
\renewcommand{\t}{\text} % text in math mode
\newcommand{\f}[2]{\dfrac{#1}{#2}} % shorthand for fractions
\newcommand{\p}[1]{\left(#1\right)} % square parenthesis
\renewcommand{\sp}[1]{\left[#1\right]} % square parenthesis
\renewcommand{\set}[1]{\left\{#1\right\}} % curly parenthesis
\renewcommand{\v}{\bm} % bold vectors
\newcommand{\uv}[1]{\hat{\v{#1}}} % unit vectors
\renewcommand{\c}{\cdot} % inner product
\newcommand{\bk}{\Braket} % shorthand for braket notation

% shorthand for frequently used commands
\renewcommand{\u}{\uparrow}
\renewcommand{\d}{\downarrow}
\newcommand{\SWAP}{\t{SWAP}}
\newcommand{\cNOT}{\t{cNOT}}
\newcommand{\NV}{\t{NV}}
\newcommand{\AC}{\t{AC}}
\newcommand{\ST}{\t{ST}}
\renewcommand{\S}{\t{S}}
\newcommand{\T}{\t{T}}
\newcommand{\R}{\mathcal R}
\newcommand{\A}{\mathcal A}
\newcommand{\B}{\mathcal B}

%%% figures
\usepackage{graphicx,grffile,float} % floats, etc.
\usepackage{multirow} % multirow entries in tables
\usepackage{footnote} % footnotes in floating objects
\usepackage[font=small,labelfont=bf]{caption} % caption text options

%%% markup
\usepackage{color,soul} % text color and other editing options
% \ul{underline}, \st{strikethrough}, and \hl{highlight}
\newcommand{\fixme}[1]{{\bf \color{red} fixme: #1}}

\begin{document}

\title{Realization of a Decoherence-Free Subspace with
  Nitrogen-Vacancy Centers and $^{13}$C Nuclear Spins}

\author{Michael A. Perlin, Zhen-Yu Wang, Jorge Casanova, and Martin
  B. Plenio}

\affiliation{Institut f\"ur Theoretische Physik, Albert-Einstein-Allee
  11, Universit\"at Ulm, D-89069 Ulm, Germany}

\begin{abstract}
  Here is the abstract.
\end{abstract}

\maketitle

\section{Introduction}

\fixme{Write about NV centers, C-13 nuclei, introduction to larmor
  pairs and DFS, quantum memory/networks, etc.}

Nitrogen-vacancy (NV) centers in diamond have been recognized in
recent years as a promising platform for quantum sensing and quantum
information technologies\cite{mamin2013nanoscale, steinert2010high,
  wang2016positioning, chou2015optimal, childress2006fault,
  yao2012scalable}. The electronic ground and optically excited states
of NV$^-$ split into spin triplets which can be polarized, detected,
and coherently manipulated with high fidelity with fields and
microwave radiation \cite{dobrovitski2013quantum}. Hyperfine coupling
between the NV center electron and $^{13}$C nuclei in bulk diamond
allows for the coherent control and measurement of individual nuclear
spins, and the realization of universal set of high-fidelity quantum
gates \cite{dobrovitski2013quantum, casanova2016noise}.

One promising application of the NV center is as a memory register for
storing quantum information. While typical control operations on the
NV electron spin happen on the scale of nanoseconds, its
room-temperature coherence time is on the scale of microseconds
\cite{dobrovitski2013quantum}. Dynamic decoupling schemes and the use
of isotopically pure diamond samples can increase the NV electron spin
coherence time by a few orders of magnitude
\cite{ryan2010robust}. Storing quantum information in the NV electron
spin itself, however, reserves a valuable resource, as the center
could otherwise be used to address nuclear spins and other quantum
processing resources. The weaker environmental coupling of $^{13}$C
nuclei present in diamond yields much longer coherence times than that
of the NV electron spin, and by entangling multiple nuclear spins,
e.g. into the singlet-triplet states
$\p{\ket{\u\d}\pm\ket{\d\u}}/\sqrt2$, it is possible to generate
logical qubits which are protected from various sources of system and
environmental dephasing noise.

In this letter, we present a means to address individual $^{13}$C
nuclear spins which are normally indistinguishable due to a symmetry
in their coupling to the NV center. We use this addressing scheme to
initialize qubits in a decoherence-free subspace (DFS) of nuclear
spins, and propose a means for realizing a coherent SWAP operation
between the NV center electron spin state and the DFS qubit.


\section{Identifying symmetrically coupled $^{13}$C nuclear spins}

Restricting the NV electron spin states to the span of
$\set{\ket0,\ket{m_s}}$ ($m_s=\pm1$), the AXY-$n$ pulse sequence
developed in by Casanova et al. in \cite{casanova2015robust} and
control scheme developed by Wang et al. in \cite{wang2016positioning}
allows addressing individual $^{13}$C nuclear spins via Hamiltonians
of the form
\begin{align}
  H_{\t{int}} = h_j^{\t{int}}\sigma_\NV^z\v I_j\c\uv n_j,
  \label{eq:H_int}
\end{align}
for a tunable coupling strength $h_j^{\t{int}}$ and direction
$\uv n_j$, where $\sigma_\NV^z\equiv\op{m_s}-\op{0}$ and $\v I_j$ is
the spin operator for nuclear spin $j$. The protocol for realizing
this effective Hamiltonian is conditional on a sufficiently unique
effective larmor frequency $\omega_j$, i.e. that for all $n\ne j$
\begin{align}
  \abs{\omega_n-\omega_j} \gg \abs{f_{k_{DD}}\v A_j^\perp},
\end{align}
where $f_{k_{DD}}$ is the fourier component of the AXY-$n$ protocol on
resonance with $\omega_j$ and $\v A_j^\perp$ ($\v A_j^\parallel$) is
the component of the hyperfine vector $\v A_j$ perpendicular
(parallel) to the effective larmor precession axis $\uv\omega_j$.

It is possible, however, for two nuclear spins (indexed by 1 and 2) to
have identical larmor frequencies $\omega_1=\omega_2\equiv\omega$,
which occurs when both nuclei lie along the edges of a cylinder
centered on the NV center and oriented along the natural NV axis. We
call such nuclei {\it larmor pairs}, and their geometry implies
$A_1^\perp=A_2^\perp\equiv A^\perp$ and
$A_1^\parallel=A_2^\parallel\equiv A^\parallel$ (see Supplementary
Information) \fixme{prove in the SI}. We wish to identify larmor
pairs and individually address the nuclei making up the pair.

The AXY-$n$ sequence and the Hamiltonian in Eq. (\ref{eq:H_int})
allows for measurement of distinct frequencies $\omega_j$, hyperfine
couplings $A_j^\perp$, and ultimately the positions of $^{13}$C nuclei
via a coupling protocol ending in an NV coherence measurement,
outlined in \cite{wang2016positioning}. When on resonance with several
frequencies $\set{\omega_j}$ which the protocol cannot resolve, the NV
coherence after a time $t$ reads
\begin{align}
  \label{eq:L}
  L = \prod_j\cos\p{\f14 f_{k_{DD}}A_j^\perp t}.
\end{align}
After selecting an effective larmor frequency $\omega$ of interest via
the scanning protocol in \cite{wang2016positioning}, coherence
measurements as per Eq. (\ref{eq:L}) for different AXY-$n$ fourier
components $f_{k_{DD}}$ (or times $t$) are sufficient to determine the
number of nuclei with effective larmor frequency $\omega$. The
protocol for measuring NV coherence can thus be used to identify a
larmor pair.

After identifying a larmor pair, in order to locate the individual
nuclei in the pair we introduce an auxiliary magnetic decoupling field
$\tilde{\v B}_{\t{dec}}\p{t} = \v
B_{\t{dec}}\cos\p{\omega_{\t{dec}}t-\phi_{\t{dec}}}$. When
$\omega_{\t{dec}}$ is on resonance with the effective larmor frequency
$\omega$ of a larmor pair (and some simple conditions on
$\v B_{\t{dec}}$ are satisfied), the coherence in Eq. (\ref{eq:L}) can
be written as
\begin{align}
  \label{eq:L_dec}
  L = \f12\p{\cos\sp{2\xi\cos\alpha_+\cos\alpha_-}
  + \cos\sp{2\xi\sin\alpha_+\sin\alpha_-}}.
\end{align}
Here $\xi$ is a constant phase determined by $f_{k_{DD}}$, $A^\perp$,
and interaction time $t$; the angle $\alpha_-$ is determined by the
directions $\uv A_j^\perp$; and $\alpha_+$ is determined by the
directions $\uv A_j^\perp$ as well as $\phi_{DD}$, $\phi_{\t{dec}}$,
and $\uv B_{\t{dec}}$. Measurement of e.g.  $L\p{\xi,\phi_{DD}}$ thus
provides sufficient information to deduce the directions
$\uv A_j^\perp$, knowing which is necessary for individual nuclear
addressing protocols (see Supplementary Info section
\ref{sec:S_locating_pairs}).


\section{Individually addressing nuclei in larmor pairs}

When trying to address one spin in a larmor pair, the protocols for
Eq. (\ref{eq:H_int}) will simultaneously couple $\sigma_\NV^z$ to both
$\v I_1\c\uv n_1$ and $\v I_2\c\uv n_2$ with identical coupling
strength ($h_j^{\t{int}}$). So long as $\uv n_1$ and $\uv n_2$
(equivalently, $\v A_1^\perp$ and $\v A_2^\perp$) are not parallel,
however, it is possible to address only one of these spins at a time
by applying additional control fields. In particular, the application
of a magnetic decoupling field
$\tilde{\v B}_{\t{dec}}\p{t}=\v B_{\t{dec}}\cos\p{\omega
  t-\phi_{\t{dec}}}$ during the AXY-$n$ addressing protocol suppresses
components of $\uv n_j$ which are perpendicular to the vector
$\v B_{\t{dec}}$ projected onto the plane perpendicular to
$\v\omega_j$ and rotated about the axis $\uv\omega_j$ by the angle
$\phi_{\t{dec}}$. One can therefore choose a direction
$\v B_{\t{dec}}$ which completely suppresses NV electron spin coupling
to spin 1 (2), leaving only coupling to spin 2 (1).

\fixme{Give more information or details about this protocol? Reference
  SI}


\section{Qubit initialization in a decoherence-free subspace}

The state space of larmor pairs contains a subspace free of
decoherence via spin coupling to the NV center electron, i.e. a
decoherence-free subspace (DFS). This DFS is spanned by the
singlet-triplet states $\p{\ket{\u\d}\pm\ket{\d\u}}/\sqrt2$, or
equivalently by $\ket{\u\d}$ and $\ket{\d\u}$. \fixme{prove that this
  subspace is decoherence-free in the SI} The protocols for addressing
an individual nucleus in a larmor pair can be used to initialize a
qubit in this DFS. In particular, one can use the gate sequences in
\cite{reiserer2016robust} to deterministically initialize the larmor
pair into $\ket{\d\d}$, then probabilistically convert this state into
$\p{\ket{\u\d}+e^{i\phi}\ket{\d\u}}/\sqrt2$ for some phase $\phi$
determined by the choice of bases for the nuclear spins. While the
sequences to initialize into $\ket{\d\d}$ require individually
addressing each nuclear spin, the conversion
$\ket{\d\d}\to\p{\ket{\u\d}+e^{i\phi}\ket{\d\u}}/\sqrt2$ can be
performed with simultaneous coupling to both larmor pairs,
i.e. without the magnetic decoupling field.

\fixme{Provide result for initialization fidelities.}


\section{Coherent SWAP operation with DFS qubit}

The protocols developed for addressing larmor pairs can be used to
perform a coherent SWAP operation between the NV center spin and the
DFS qubit. In principle, such a coherent SWAP allows using larmor
pairs as quantum memory registers.

\fixme{Expand.}


\section{Conclusions}


\section{Acknowledgments}


\bibliography{\jobname}

\pagebreak
\clearpage
\widetext \makeatletter
\begin{center}
  \large \bf Supplementary Information for ``\@title''
\end{center}
\setcounter{equation}{0}
\setcounter{figure}{0}
\setcounter{table}{0}
\renewcommand{\theequation}{S\arabic{equation}}
\renewcommand{\thefigure}{S\arabic{figure}}
\renewcommand{\bibnumfmt}[1]{[S#1]}
\renewcommand{\citenumfont}[1]{S#1}

\setcounter{section}{0}
\renewcommand{\thesection}{S.I. \Roman{section}}

\tableofcontents{}

\section{Basic system model}

The Hamiltonian of the NV center system in the presence of a strong,
static magnetic field $\v B$ can be decomposed as
\begin{align}
  H_{\t{sys}} = H_\NV^{\t{GS}} + H_{\t{hf}} + H_{\t{nZ}}  + H_{\t{nn}},
\end{align}
where
\begin{align}
  H_\NV^{\t{GS}} = D\p{\v S\c\uv z}^2 - \gamma_e\v B\c\v S
\end{align}
is the ground-state NV electron spin Hamiltonian with zero-field
splitting $D$, spin-1 operator $\v S$, natural NV axis $\uv z$, and
electron gyromagnetic ratio $\gamma_e$;
\begin{align}
  H_{\t{hf}} = \sum_j\f{\gamma_e\gamma_j}{4\pi r_j^3}
  \p{\v S\c\v I_j-3\sp{\v S\c\uv r_j}\sp{\v I_j\c\uv r_j}}
\end{align}
is the hyperfine coupling between the NV electron spin (located at the
origin) and nuclei indexed by $j$ located at $\v r_j$;
\begin{align}
  H_{\t{nZ}} = -\sum_j\gamma_j\v B\c\v I_j
\end{align}
is the Zeeman Hamiltonian for nuclei indexed by $j$ with gyromagnetic
ratio $\gamma_j$ and spin operators $\v I_j$; and
\begin{align}
  H_{\t{nn}} = \sum_{j<k}\f{\gamma_j\gamma_k}{4\pi r_{jk}^3}
  \p{\v I_j\c\v I_k-3\sp{\v I_j\c\uv r_{jk}}\sp{\v I_k\c\uv r_{jk}}}
\end{align}
is the internuclear coupling with $\v r_{jk}\equiv\v r_j-\v r_k$. We
will hereafter neglect internuclear coupling in all analytical work,
though we will keep $H_{\t{nn}}$ in any numerical results.

Changing into the rotating frame of $H_\NV^{\t{GS}}$ affects only the
hyperfine Hamiltonian, which becomes
\begin{align}
  \tilde H_{\t{hf}} = S_z\sum_j \v A_j\c\v I_j,
\end{align}
where we define the hyperfine field
\begin{align}
  \v A_j = \f{\gamma_e\gamma_j}{4\pi r_j^3}\p{\uv z - 3\sp{\uv
  r_j\c\uv z}\uv r_j}.
\end{align}
The system Hamiltonian is thus
\begin{align}
  \tilde H_{\t{sys}} = S_z\sum_j \v A_j\c\v I_j
  -\sum_j\gamma_j\v B\c\v I_j.
\end{align}

When NV electron spin is restricted to the span of
$\set{\ket0,\ket{m_s}}$ with $m_s\in\set{1,-1}$,
\begin{align}
  S_z = \f{m_s}2\p{\sigma_\NV^z+\mathbf1},
\end{align}
where $\sigma_\NV^z\equiv\op{m_s}-\op{0}$ and
$\mathbf1\equiv\op{m_s}+\op{0}$. Defining the interaction Hamiltonian
\begin{align}
  H_{\t{int}} = \f12m_s\sigma_\NV^z\sum_j\v A_j\c\v I_j
\end{align}
the effective larmor frequency
\begin{align}
  \v\omega_j = \gamma_j\v B - \f{m_s}2\v A_j,
\end{align}
and the effective Zeeman Hamiltonian
\begin{align}
  H_{\t{nZ}}^{\t{eff}} = -\sum_j\p{\gamma_j\v B - \f{m_s}2\v A_j}\c\v I_j
  = -\sum_j\v\omega_j\c\v I_j,
\end{align}
the system Hamiltonian can be written as
\begin{align}
  \tilde H_{\t{sys}} = H_{\t{int}} + H_{\t{nZ}}^{\t{eff}}.
  \label{eq:S_H_sys_int_nZ}
\end{align}

Moving into the rotating frame of $H_{\t{nZ}}^{\t{eff}}$, the entire
system is described by the Hamiltonian
\begin{align}
  \tilde H_{\t{int}} = \f12m_s\sigma_\NV^z
  \sum_j\v\A_j\p{t}\c\v I_j,
  \label{eq:S_H_int_no_DD}
\end{align}
where $\v\A_j\p{t}$ is the vector $\v A_j$ rotated about $\uv\omega_j$
by an angle $\omega_jt$. Written explicitly,
\begin{align}
  \v\A_j\p{t} = \v A_j^\parallel + \v A_j^\perp\cos\p{\omega_jt}
  + \v A_j^{\perp\perp}\sin\p{\omega_jt},
  \label{eq:S_A_rot}
\end{align}
where
\begin{align}
  \v A_j^\parallel\equiv\p{\v A_j\c\uv\omega_j}\uv\omega_j
\end{align}
is the projection of $\v A_j$ along $\v \omega_j$,
\begin{align}
  \v A_j^\perp\equiv\v A_j-\v A_j^\parallel
\end{align}
is the projection of $\v A_j$ onto the plane orthogonal to
$\v\omega_j$, and
\begin{align}
  \v A_j^{\perp\perp}\equiv\uv\omega_j\times\v A_j^\perp
\end{align}
is the vector $\v A_j^\perp$ rotated about $\uv\omega_j$ by $\pi/2$.


\section{Dynamic decoupling and nuclear spin addressing}
\label{sec:S_dynamic_decoupling}

Dynamic decoupling schemes effectively prepend a modulation function
\begin{align}
  F\p{t}=\sum_kf_k\cos\p{k\omega_{DD}t-\phi_{DD}}
\end{align}
with fundamental frequency $\omega_{DD}$, phase $\phi_{DD}$, and range
$\set{1,-1}$ to $\sigma_\NV^z$, so that the system Hamiltonian reads
\begin{align}
  \tilde H_{\t{int}} =
  \f12m_sF\p{t}\sigma_\NV^z\sum_j\v\A_j\p{t}\c\v I_j.
\end{align}
At a resonance $k_{DD}\omega_{DD}\equiv\omega_r\gg A_j$, the secular
approximation allows one to neglect $\v A_j^\parallel$ and express
\begin{align}
  \tilde H_{\t{int}} = \f14 f_{k_{DD}} m_s\sigma_\NV^z
  \sum_{j:\omega_j=\omega_r} A_j^\perp\v I_j\c\uv n_j,
  \label{eq:S_H_int_resonance}
\end{align}
where $\uv n_j$ is a unit vector in the direction of $\v A_j^\perp$
rotated about $\uv\omega_j$ by $\phi_{DD}$, i.e.
\begin{align}
  \uv n_j = \uv A_j^\perp\cos\phi_{DD} +
  \uv A_j^{\perp\perp}\sin\phi_{DD}.
\end{align}
One can concisely write Eq. (\ref{eq:S_H_int_resonance}) with
\begin{align}
  h_j^{\t{int}}\equiv f_{k_{DD}}m_sA_j^\perp/4
\end{align}
as
\begin{align}
  \tilde H_{\t{int}} = \sum_{j:\omega_j=\omega_r}
  h_j^{\t{int}}\sigma_\NV^z\v I_j\c\uv n_j.
  \label{eq:S_H_int}
\end{align}

When the harmonics of $F\p{t}$ are off resonance with all $\omega_j$
and $\omega_{DD}\gg A_j$ for all $j$, the secular approximation allows
one to neglect $\tilde H_{\t{int}}$ entirely. In this case, one can
apply an auxiliary magnetic field
$\tilde{\v B}_{\t{ctl}}\p{t}=\v
B_{\t{ctl}}\cos\p{\omega_{\t{ctl}}t-\phi_{\t{ctl}}}$ and shift into
the rotating frame of $H_{\t{nZ}}^{\t{eff}}$, much like the frame
shift between Eqs. (\ref{eq:S_H_sys_int_nZ}) and
(\ref{eq:S_H_int_no_DD}), to realize the Hamiltonian
\begin{align}
  \tilde H_{\t{sys}}^{\t{ctl}} =
  -\sum_j\gamma_j\v\B_{\t{ctl},j}\p{t}
  \cos\p{\omega_{\t{ctl}}t-\phi_{\t{ctl}}}\c\v I_j,
\end{align}
with $\v\B_{\t{ctl},j}\p{t}$ defined using $\v B_{\t{ctl}}$ and
$\v\omega_j$ the same way that $\v\A_j\p{t}$ was defined using
$\v A_j$ and $\v\omega_j$ in Eq. (\ref{eq:S_A_rot}), i.e. as
$\v B_{\t{ctl}}$ rotated about $\uv\omega_j$ by an angle $\omega_jt$:
\begin{align}
  \v\B_{\t{ctl},j}\p{t} = \v B_{\t{ctl},j}^\parallel
  + \v B_{\t{ctl},j}^\perp\cos\p{\omega_jt}
  + \v B_{\t{ctl},j}^{\perp\perp}\sin\p{\omega_jt},
\end{align}
where
\begin{align}
  \v B_{\t{ctl},j}^\parallel
  \equiv \p{\v B_{\t{ctl}} \c\uv\omega_j}\uv\omega_j,
  && \v B_{\t{ctl},j}^\perp
  \equiv \v B_{\t{ctl}} - \v B_{\t{ctl},j}^\parallel,
  && \v B_{\t{ctl},j}^{\perp\perp}
  \equiv \uv\omega_j\times\v B_{\t{ctl}}^\perp.
\end{align}

When $\omega_{\t{ctl}}\gg\gamma_jB_{\t{ctl}}$, the secular
approximation allows one to neglect $\v B_{\t{ctl},j}^\parallel$ and
express
\begin{align}
  \tilde H_{\t{ctl}} = -\f12\sum_{j:\omega_j=\omega_{\t{ctl}}}
  \gamma_jB_{\t{ctl},j}^\perp \v I_j\c\uv m_j,
  \label{eq:S_H_ctl_resonance}
\end{align}
where $\uv m_j$ is the direction of $\v B_{\t{ctl},j}^\perp$ rotated
about $\uv\omega_j$ by $\phi_{\t{ctl}}$, i.e.
\begin{align}
  \uv m_j = \uv B_{\t{ctl},j}^\perp\cos\phi_{\t{ctl}} +
  \uv B_{\t{ctl},j}^{\perp\perp}\sin\phi_{\t{ctl}}.
\end{align}
One can concisely write Eq. (\ref{eq:S_H_ctl_resonance}) with
\begin{align}
  h_j^{\t{ctl}}\equiv-\gamma_jB_{\t{ctl},j}^\perp/2
\end{align}
as
\begin{align}
  \tilde H_{\t{ctl}} = \sum_{j:\omega_j=\omega_{\t{ctl}}}
  h_j^{\t{ctl}}\v I_j\c\uv m_j.
  \label{eq:S_H_ctl}
\end{align}

In both Eqs. (\ref{eq:S_H_int}) and (\ref{eq:S_H_ctl}), one can choose
$\phi_{DD}$ and $\phi_{\t{ctl}}$ such that $\uv n_j\p\phi$ and
$\uv m_j\p{\phi_{\t{ctl}}}$ lie anywhere in the plane perpendicular to
$\uv\omega_j$. One can also tune $h_j^{\t{int}}$ and $h_j^{\t{ctl}}$
respectively via free parameters in the AXY-$n$ sequence and
$\v B_{\t{ctl}}$. These controls, together with rotations of the NV
electron spin, allow one to realize a universal set of quantum
gates. An exception to this statement is when two or more nuclei have
the same effective larmor frequency $\omega_j$, in which case they are
addressed simultaneously in Eqs. (\ref{eq:S_H_int}) and
(\ref{eq:S_H_ctl}).


\section{Addressing larmor pairs}
\label{sec:S_addressing_pairs}

We call two spins indexed by $j$ and $k$ elements of a \emph{larmor
  set} if their effective larmor frequencies are equal,
i.e. $\omega_j=\omega_k$, and refer to a larmor set with exactly two
spins as a \emph{larmor pair}. The problem with individually
addressing elements of a larmor set is manifest in
Eqs. (\ref{eq:S_H_int}) and (\ref{eq:S_H_ctl}), wherein the previously
developed procedure for nuclear spin addressing does not single out
any nuclei. Nonetheless, under certain conditions it is possible to
individually address elements of a larmor set by applying additional
control fields.

When employing a dynamic decoupling protocol on resonance with an
effective larmor frequency $\omega_r$, one can apply an additional
magnetic decoupling field
$\tilde{\v B}_{\t{dec}}\p{t}=\v B_{\t{dec}}\cos\p{\omega_r
  t-\phi_{\t{dec}}}$ and realize, in the frame of $H_\NV^{\t{GS}}$,
the system Hamiltonian
\begin{align}
  \tilde H_{\t{sys}}^{\t{dec}}
  = H_{\t{int}} + H_{\t{nZ}}^{\t{eff}} + H_{\t{dec}},
\end{align}
where
\begin{align}
  H_{\t{dec}} = -\sum_j\gamma_j\v B_{\t{dec}}\c\v I_j
  \cos\p{\omega_rt-\phi_{\t{dec}}}.
\end{align}

In the frame of $H_{\t{nZ}}^{\t{eff}}=-\sum_j\v\omega_j\c\v I_j$, the
decoupling Hamiltonian becomes
\begin{align}
  \tilde H_{\t{dec}} = -\sum_j\gamma_j\v\B_{\t{dec},j}
  \c\v I_j \cos\p{\omega_rt-\phi_{\t{dec}}}
  \label{eq:S_H_dec_eff_full}
\end{align}
with
\begin{align}
  \v\B_{\t{dec},j} = \v B_{\t{dec},j}^\parallel
  + \v B_{\t{dec},j}^\perp\cos\p{\omega_j t}
  + \v B_{\t{dec},j}^{\perp\perp}\sin\p{\omega_j t}
\end{align}
and
\begin{align}
  \v B_{\t{dec},j}^\parallel
  \equiv \p{\v B_{\t{dec}}\c\uv\omega_j}\uv\omega_j,
  && \v B_{\t{dec},j}^\perp
  \equiv \v B_{\t{dec}} - \v B_{\t{dec},j}^\parallel,
  && \v B_{\t{dec},j}^{\perp\perp}
  \equiv \uv\omega_j\times\v B_{\t{dec}}^\perp.
\end{align}
If $\omega_r,\omega_j\gg \gamma_jB_{\t{dec}}$, then we can neglect
$\v B_{\t{dec},j}^\parallel$ in Eq. (\ref{eq:S_H_dec_eff_full}) by the
secular approximation, and by the rotating approximation express
\begin{align}
  \tilde H_{\t{dec}} = -\sum_{j:\omega_j=\omega_r}
  \f12\gamma_j\p{\v B_{\t{dec},j}^\perp\cos\phi_{\t{dec}} +
  \v B_{\t{dec},j}^{\perp\perp}\sin\phi_{\t{dec}}}\c\v I_j.
\end{align}
Defining
\begin{align}
  \v\nu_j = \f12\gamma_j\p{\v B_{\t{dec},j}^\perp\cos\phi_{\t{dec}}
  + \v B_{\t{dec},j}^{\perp\perp}\sin\phi_{\t{dec}}},
\end{align}
we can express
\begin{align}
  \tilde H_{\t{dec}} = -\sum_{j:\omega_j=\omega_r}\v\nu_j\c\v I_j.
\end{align}
The system Hamiltonian in the frame of $H_{\t{nZ}}^{\t{eff}}$ thus
takes the form
\begin{align}
  \tilde H_{\t{sys}}^{\t{dec}} = \tilde H_{\t{int}} + \tilde H_{\t{dec}}
  = \sum_{j:\omega_j=\omega_r}
  \p{h_j^{\t{int}}\sigma_\NV^z\v I_j\c\uv n_j - \v\nu_j\c\v I_j}.
\end{align}

In the frame of $\tilde H_{\t{dec}}$, the entire system is governed by
the Hamiltonian
\begin{align}
  \tilde H_{\t{int}}^{\t{dec}} = \sum_{j:\omega_j=\omega_r}
  h_j^{\t{int}}\sigma_\NV^z\v I_j\c\tilde{\v n}_j\p{t},
  \label{eq:S_H_int_dec_full}
\end{align}
where $\tilde{\v n}_j\p{t}$ is the unit vector $\uv n_j$ rotated about
$\uv\nu_j$ by $\nu_jt$, i.e.
\begin{align}
  \tilde{\v n}_j\p{t} = \v n_j^\parallel +
  \v n_j^\perp\cos\p{\nu_jt} + \v n_j^{\perp\perp}\sin\p{\nu_jt},
\end{align}
with
\begin{align}
  \v n_j^\parallel &\equiv \p{\uv n_j\c\uv\nu_j}\uv\nu_j, \\
  \v n_j^\perp &\equiv \uv n_j - \uv n_j^\parallel, \\
  \v n_j^{\perp\perp} &\equiv \uv\nu_j\times\uv n_j.
\end{align}
If $\omega_j\gg\nu_j\gg h_j^{\t{int}}$, the secular approximation
allows us to neglect $\v n_j^\perp$ and $\v n_j^{\perp\perp}$ in
Eq. (\ref{eq:S_H_int_dec_full}), leaving
\begin{align}
  \tilde H_{\t{int}}^{\t{dec}}
  = \sum_{j:\omega_j=\omega_r}
  \tilde h_j^{\t{int}}\sigma_\NV^z\v I_j\c\uv\nu_j,
  \label{eq:S_H_int_dec}
\end{align}
where
\begin{align}
  \tilde h_j^{\t{int}}
  = h_j^{\t{int}}\uv n_j\c\uv\nu_j.
\end{align}
Letting $\theta_j$ and $\theta_{\t{dec}}$ respectively be the angular
positions of $\uv A_j^\perp$ and $\uv B_{\t{dec}}^\perp$ in the plane
perpendicular to $\uv\omega_j$, we can say that
\begin{align}
  \tilde h_j^{\t{int}} = h_j^{\t{int}}\cos\p{\theta_j-\eta},
\end{align}
where $\eta=-\phi_{DD}+\theta_{\t{dec}}+\phi_{\t{dec}}$.

In words, the decoupling field $\tilde{\v B}_{\t{dec}}\p{t}$ modifies
$\tilde H_{\t{int}}$ in Eq. (\ref{eq:S_H_int}) by suppressing the
components of $\uv n_j$ (i.e. $\uv A_j^\perp$ rotated about
$\uv\omega_j$ by $\phi_{DD}$) which are perpendicular to the
respective $\uv\nu_j$ (i.e.  $\uv B_{\t{dec},j}^\perp$ rotated about
$\uv\omega_j$ by $\phi_{\t{dec}}$). The utility of this decoupling
field lies in the fact that when $\uv n_j$ and $\uv\nu_j$ are
orthogonal, the interaction with spin $j$ is suppressed as
$\tilde h_j^{\t{int}}=0$. Considering a single larmor pair, one can
therefore apply a decoupling field to eliminate one of the two terms
in Eq. (\ref{eq:S_H_int_dec}) and realize an effective Hamiltonian
which couples the NV electron spin to only one nuclear spin. This
scheme fails for larmor pairs with mutually parallel $\v A_j^\perp$.

The tools developed in this paper do not allow for single-spin
operations on elements of a larmor pair. Such operations must
therefore be either forgone, or else realized via composite operations
making use of Eqs. (\ref{eq:S_H_ctl}) and (\ref{eq:S_H_int_dec}). The
same is true of spin-spin coupling between the NV electron spin and a
single element of a larmor set with more than two nuclei.


\section{Identifying and locating larmor pairs}
\label{sec:S_locating_pairs}

Larmor pairs can be identified and located via NV coherence
measurements as in \cite{wang2016positioning}. When the AXY-$n$
sequence is on resonance with a larmor pair, the NV coherence reads
\begin{align}
  L = \cos^2\xi,
\end{align}
where the phase $\xi = h^{\t{int}}t = f_{k_{DD}}m_sA_j^\perp t/4$.
Applying a decoupling field as in section \ref{sec:S_addressing_pairs}
modifies $h_j^{\t{int}}\to h_j^{\t{int}}\cos\p{\theta_j-\eta}$, which
makes the NV coherence
($\eta=-\phi_{DD}+\theta_{\t{dec}}+\phi_{\t{dec}}$)
\begin{align}
  L
  = \cos\p{\xi\cos\sp{\theta_1-\eta}}\cos\p{\xi\cos\sp{\theta_2-\eta}}.
\end{align}
Defining
\begin{align}
  \alpha_- = \f12\p{\sp{\theta_1-\eta}-\sp{\theta_2-\eta}}
  = \f12\p{\theta_1-\theta_2},
  && \alpha_+ = \f12\p{\sp{\theta_1-\eta}+\sp{\theta_2-\eta}}
     = \f12\p{\theta_1+\theta_2} - \eta,
\end{align}
we can expand
\begin{align}
  L
  = \f12\p{\cos\sp{2\xi\cos\alpha_+\cos\alpha_-}
  + \cos\sp{2\xi\sin\alpha_+\sin\alpha_-}}.
\end{align}
When $\alpha_+$ is an integer multiple of $\pi/2$, this coherence
simplifies to ($n\in\mathbb Z$)
\begin{align}
  \label{eq:S_L_simplified}
  L\p{\alpha_+=n\pi} = \f12\p{1 + \cos\sp{2\xi\cos\alpha_-}},
  && L\p{\alpha_+=n\pi+\f\pi2}
     = \f12\p{1 + \cos\sp{2\xi\sin\alpha_-}}.
\end{align}
By collecting data on $L\p{\xi,\eta}$, one can identify fixed angles
$\bar\eta$ at which $L_{\bar\eta}\p\xi\equiv L\p{\xi,\bar\eta}$ takes
the form above. Knowing $L_{\bar\eta}\p\xi$ is then sufficient to
deduce the values of $\theta_1$ and $\theta_2$ up to a degeneracy
which does not affect nuclear spin addressing protocols. To find the
angles $\set{\theta_j}$ and understand this degeneracy, we first note
that changing these angles by $\pi$ as $\theta_j\to\theta_j+\pi$ does
not affect any protocols, so we can safely enforce all
$\theta_j\in\sp{0,\pi}$ and in turn $\alpha_\pm\in\sp{0,\pi}$. The
coherence $L\p{\xi,\eta}$ then takes the forms in
(\ref{eq:S_L_simplified}) when $\alpha_+\in\set{0,\pi/2}$, in which
case a fourier transform of $L_{\bar\eta}\p\xi$ can be used to
identify the coherence modes
\begin{align}
  \omega_L\p{\alpha_+=0}=2\xi\cos\alpha_-,
  && \omega_L\p{\alpha_+=\pi/2}=2\xi\sin\alpha_-,
\end{align}
and in turn the angles
\begin{align}
  \alpha_-\p{\alpha_+=0} = \acos\p{\f1{2\xi}\omega_L\p{\alpha_+=0}},
  && \alpha_-\p{\alpha_+=\f\pi2}
     = \asin\p{\f1{2\xi}\omega_L\p{\alpha_+=\f\pi2}}.
\end{align}
When $\alpha_+=0$, the angles $\theta_1$ and $\theta_2$ are then
\begin{align}
  \label{eq:S_nuclear_angles}
  \theta_1 = \bar\eta\p{\alpha_+=0} + \alpha_-\p{\alpha_+=0},
  && \theta_2 = \bar\eta\p{\alpha_+=0} - \alpha_-\p{\alpha_+=0},
\end{align}
The obvious problem with this recipe for $\set{\theta_j}$ is the fact
that measuring $L_{\bar\eta}\p\xi$ does not allow one to distinguish
the case of $\alpha_+=0$ from $\alpha_+=\pi/2$, as the form of
$L_{\bar\eta}\p\xi$ is identical in both cases. As it turns out, in
finding $\theta_1$ and $\theta_2$ via the prescription above, one is
free to simply assume without consequence that $\alpha_+=0$ whenever
the coherence $L_{\bar\eta}\p\xi$ takes the form in
(\ref{eq:S_L_simplified}). Incorrectly identifying the value of
$\alpha_+'=\pi/2-\alpha_+$ instead of $\alpha_+$ will result in the
deduction of $\alpha_-'=\pi/2-\alpha_-$ instead of $\alpha_-$. The
prescriptions in (\ref{eq:S_nuclear_angles}) then yield
\begin{align}
  \bar\eta\p{\alpha_+'=0} + \alpha_-'\p{\alpha_+'=0}
  = \sp{\bar\eta\p{\alpha_+=0} - \f\pi2}
  + \sp{\f\pi2 - \alpha_-\p{\alpha_+=0}}
  = \bar\eta\p{\alpha_+=0} - \alpha_-\p{\alpha_+=0} = \theta_2
\end{align}
and similarly
\begin{align}
  \bar\eta\p{\alpha_+'=0} - \alpha_-'\p{\alpha_+'=0} = \theta_1.
\end{align}
Using $\alpha_\pm'$ instead of $\alpha_\pm$ to find $\set{\theta_j}$
thus amounts to merely relabeling nuclei, which has no physical
consequence.


\section{Coherent SWAP with a singlet-triplet qubit}

In the following section, we will consider an addressable larmor pair
with effective larmor frequencies $\omega$. We will use the basis
$\set{\ket\u\equiv\ket 0, \ket\d\equiv\ket{m_s}}$ for the spin states
of the NV electron, and the standard basis
$\set{\ket{\u\u}, \ket{\u\d}, \ket{\d\u}, \ket{\d\d}}$ for the spin
states of the nuclei. We will refer to the logical qubit consisting of
the basis states $\set{\ket{\u\d},\ket{\d\u}}$ as the anti-correlated
(AC) qubit, and the logical qubit consisting of the basis states
$\set{\ket\S,\ket\T}$ as the singlet-triplet (ST) qubit, where
\begin{align}
  \ket\S \equiv \f1{\sqrt2}\p{\ket{\u\d}-\ket{\d\u}},
  &&  \ket\T \equiv \f1{\sqrt2}\p{\ket{\u\d}+\ket{\d\u}}.
\end{align}

A SWAP operation between two qubits indexed by 1 and 2 can be realized
via the composite operation
\begin{align}
  \SWAP^{1,2} = \cNOT^{2\to1}\cNOT^{1\to2}\cNOT^{2\to1},
\end{align}
where $\cNOT^{j\to k}$ is the controlled-NOT gate, which flips the
state of qubit $k$ conditionally on the state of qubit $j$. The
$\cNOT^{j\to k}$ gate is realized via
\begin{align}
  \cNOT^{j\to k} = \exp\p{-j\f\pi2I_j^z} \exp\p{-j\f\pi2I_k^x}
  \exp\p{j\pi I_j^z I_k^x}.
\end{align}

If we wish to perform a SWAP operation between the NV electron spin
and an ST qubit, it is easier to first realize the operation
\begin{align}
  \SWAP^{\NV,\AC}
  = {\cNOT^{\AC\to\NV}}^\dagger\cNOT^{\NV\to\AC}\cNOT^{\AC\to\NV},
  \label{eq:S_SWAP_NVAC_sketch}
\end{align}
where
\begin{align}
  \cNOT^{\AC\to\NV} = \cNOT^{1\to\NV},
  && \cNOT^{\NV\to\AC} = \cNOT^{\NV\to2}\cNOT^{\NV\to1}.
\end{align}
For later convenience, Eq. (\ref{eq:S_SWAP_NVAC_sketch}) makes use of
the fact that ${\cNOT^{j\to k}}^\dagger=\cNOT^{j\to k}$. Therefore
(with $\v I_\NV=\v\sigma_\NV/2$),
\begin{align}
  \SWAP^{\NV,\AC}
  &= \exp\p{-j\f\pi2 I_\NV^x} \exp\p{j\f\pi2 I_1^{z'}}
    \exp\p{-j\f\pi2 \sigma_\NV^x I_1^{z'}} \tag*{} \\
  &~~\times \exp\p{-j\f\pi2 I_\NV^z} \exp\p{-j\f\pi2 I_1^{x'}}
    \exp\p{j\f\pi2 \sigma_\NV^z I_1^{x'}} \tag*{} \\
  &~~\times \exp\p{-j\f\pi2 I_\NV^z} \exp\p{-j\f\pi2 I_2^{x''}}
    \exp\p{j\f\pi2 \sigma_\NV^z I_2^{x''}} \tag*{} \\
  &~~\times \exp\p{-j\f\pi2 I_\NV^x} \exp\p{-j\f\pi2 I_1^{z'}}
    \exp\p{j\f\pi2 \sigma_\NV^x I_1^{z'}},
    \label{eq:S_SWAP_NVAC_full}
\end{align}
where we make explicit the fact that the the spin of nuclei 1 and 2
may be given in their own respective bases
$\set{\uv x',\uv y',\uv z'}$ and $\set{\uv x'',\uv y'',\uv z''}$. We
can evaluate the first and last rotations of spin 1 in
Eq. (\ref{eq:S_SWAP_NVAC_full}) and combine factors to get
\begin{align}
  \SWAP^{\NV,\AC}
  &= \exp\p{j\f\pi2 I_\NV^x} \exp\p{-j\f\pi2 \sigma_\NV^x I_1^{z'}}
    \tag*{} \\
  &~~\times \exp\p{-j\pi I_\NV^z} \exp\p{j\f\pi2\sp{I_1^{y'}-I_2^{x''}}}
    \tag*{} \\
  &~~\times \exp\p{-j\f\pi2 \sigma_\NV^z I_1^{y'}}
    \exp\p{j\f\pi2 \sigma_\NV^z I_2^{x''}} \tag*{} \\
  &~~\times \exp\p{j\f\pi2 \sigma_\NV^x I_1^{z'}}
    \exp\p{-j\f\pi2 I_\NV^x}.
\end{align}
Letting
\begin{align}
  \uv x' = \uv y'' &= \uv A_1^\perp, \\
  \uv y' = -\uv x'' &= \uv A_1^{\perp\perp}, \\
  \uv z' = \uv z'' &= \uv\omega_1\approx\uv z, \label{eq:S_w=z}
\end{align}
where the approximation in Eq. (\ref{eq:S_w=z}) holds for
$\gamma_1B\gg A_1/2$,
\begin{align}
  \SWAP^{\NV,\AC}
  &= \exp\p{j\f\pi2 I_\NV^x}
    \exp\p{-j\f\pi2 \sigma_\NV^x I_1^z} \tag*{} \\
  &~~\times \exp\p{-j\pi I_\NV^z}
    \exp\p{j\f\pi2\sum_jI_j^{y'}} \tag*{} \\
  &~~\times \exp\p{-j\f\pi2 \sigma_\NV^z I_1^{y'}}
    \exp\p{-j\f\pi2 \sigma_\NV^z I_2^{y'}} \tag*{} \\
  &~~\times \exp\p{j\f\pi2 \sigma_\NV^x I_1^z} \exp\p{-j\f\pi2 I_\NV^x}.
    \label{eq:S_SWAP_NVAC_sxsz}
\end{align}
As we cannot directly realize the coupling $\sigma_x^\NV I_z^1$, we
must perform rotations as
\begin{align}
  \exp\p{-j\f\pi2 \sigma_\NV^x I_1^z}
  = \exp\p{-j\f\pi2 I_\NV^y} \exp\p{j\f\pi2 I_1^{y'}}
  \exp\p{-j\f\pi2 \sigma_\NV^z I_1^{x'}}
  \exp\p{-j\f\pi2 I_1^{y'}} \exp\p{j\f\pi2 I_\NV^y}.
\end{align}
Substituting this identity into Eq. (\ref{eq:S_SWAP_NVAC_sxsz}) yields
\begin{align}
  \SWAP^{\NV,\AC}
  &= \exp\p{j\f\pi2 I_\NV^x} \exp\p{-j\f\pi2 I_\NV^y} \tag*{} \\
  &~~\times \exp\p{j\f\pi2\sum_jI_j^{y'}}
    \exp\p{-j\f\pi2 \sigma_\NV^z I_1^{x'}}
    \exp\p{j\f\pi2 I_\NV^y} \tag*{} \\
  &~~\times \exp\p{-j\pi I_\NV^z}
    \exp\p{j\f\pi2\sum_jI_j^{y'}} \tag*{} \\
  &~~\times \exp\p{-j\f\pi2 \sigma_\NV^z I_1^{y'}}
    \exp\p{-j\f\pi2 \sigma_\NV^z I_2^{y'}} \tag*{} \\
  &~~\times \exp\p{-j\f\pi2 I_\NV^y}
    \exp\p{j\f\pi2 \sigma_\NV^z I_1^{x'}}
    \exp\p{-j\f\pi2\sum_jI_j^{y'}} \tag*{} \\
  &~~\times \exp\p{j\f\pi2 I_\NV^y} \exp\p{-j\f\pi2 I_\NV^x},
    \label{eq:S_SWAP_NVAC}
\end{align}
where we have canceled out two rotations of the form
$\exp\p{\pm j\pi/2~I_1^{y'}}$ and inserted two rotations of the form
$\exp\p{\pm j\pi/2~I_2^{y'}}$ with no net effect.

The $\SWAP^{\NV,\ST}$ can be realized using the $\SWAP^{\NV,\AC}$ gate
via
\begin{align}
  \SWAP^{\NV,\ST} = R_\NV^\dagger \SWAP^{\NV,\AC} R_\NV,
\end{align}
where the rotation $R_\NV$ takes $\ket\u\to\p{\ket\u-\ket\d}/\sqrt2$
and $\ket\d\to\p{\ket\u+\ket\d}/\sqrt2$. Written explicitly,
\begin{align}
  R_\NV = \exp\p{j\f\pi2 I_\NV^y}.
\end{align}
We therefore have that
\begin{align}
  \SWAP^{\NV,\ST}
  &= \exp\p{j\pi\v I_\NV\c\uv a} \tag*{} \\
  &~~\times \exp\p{j\f\pi2\sum_jI_j^{y'}}
    \exp\p{-j\f\pi2 \sigma_\NV^z I_1^{x'}}
    \exp\p{j\f\pi2 I_\NV^y} \tag*{} \\
  &~~\times \exp\p{-j\pi I_\NV^z}
    \exp\p{j\f\pi2\sum_jI_j^{y'}} \tag*{} \\
  &~~\times \exp\p{-j\f\pi2 \sigma_\NV^z I_1^{y'}}
    \exp\p{-j\f\pi2 \sigma_\NV^z I_2^{y'}} \tag*{} \\
  &~~\times \exp\p{-j\f\pi2 I_\NV^y}
    \exp\p{j\f\pi2 \sigma_\NV^z I_1^{x'}}
    \exp\p{-j\f\pi2\sum_jI_j^{y'}} \tag*{} \\
  &~~\times \exp\p{-j\pi\v I_\NV\c\uv a},
    \label{eq:S_SWAP_NVST}
\end{align}
where $\v a = \uv x - \uv y$ and we have used the fact that
\begin{align}
  \exp\p{j\f\pi2 I^y} \exp\p{-j\f\pi2 I^x}
  \exp\p{j\f\pi2 I^y}
  = \exp\p{-j\pi\v I\c\uv a}.
\end{align}

In the subspace of interest, we have thus constructed the
non-entangling SWAP operation
\begin{align}
  \SWAP^{\NV,\ST} = \op{\u\S}{\S\u} + \op{\u\T}{\d\S}
  + \op{\d\S}{\u\T} + \op{\d\T}{\d\T}.
\end{align}
If $\gamma_1=\gamma_2$, i.e. if nuclei 1 and 2 are of the same
species, all factors in Eq. (\ref{eq:S_SWAP_NVST}) can be realized
directly via the protocols for Eq. (\ref{eq:S_H_ctl}) and
Eq. (\ref{eq:S_H_int_dec}), as well as rotations of the NV electron
spin.

As a last note, it is possible to realize $\SWAP^{\NV,\ST}$ via cNOT
operations between the NV and ST qubits directly, i.e. without
performing $\SWAP^{\NV,\AC}$, but doing so requires performing many
more individual spin addressing operations, which results in a
substantially lower fidelity.

\fixme{when we wish to read out the ST qubit, it is possible to do so
  without performing the full coherent SWAP operation so long as we
  are willing to discard the initial state of the NV center. Talk
  about this.}


\section{Robustness against noise}

\fixme{Talk about the robustness of the ST qubit against dephasing via
  the NV center, as well as its robustness against slowly varying
  (frequencies $\nu\ll\gamma_jB$) external magnetic noise.}


\section{Abundance of larmor pairs}

For any given hyperfine cutoff $A_{\t{min}}$ and an isotopic abundance
$n$ of $^{13}$C, we can compute exactly the probability
$P\p{A_{\t{min}},n}$ of finding an addressable larmor pair for which
$A_j\ge A_{\t{min}}$. The simulations developed for this paper can
also be used to test the computation of $P\p{A_{\t{min}},n}$ via the
Monte Carlo method of simply generating many NV systems and counting
the proportion of them which contain an addressable larmor pair.


\section{SWAP fidelities via AXY-$n$}

\fixme{Discuss simulations and any relevant nuances therein. Present
  histograms of $\SWAP^{\NV,\ST}$ fidelities and net operation times.}


\bibliography{\jobname}

\end{document}
