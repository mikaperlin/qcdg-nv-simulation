\documentclass[twocolumn]{revtex4-1}

% \usepackage{showframe}

\usepackage{hyperref} % for linking references

%%% math, symbols, etc.
\usepackage{physics,braket,bm}
\renewcommand{\t}{\text} % text in math mode
\newcommand{\f}[2]{\dfrac{#1}{#2}} % shorthand for fractions
\newcommand{\p}[1]{\left(#1\right)} % square parenthesis
\renewcommand{\sp}[1]{\left[#1\right]} % square parenthesis
\renewcommand{\set}[1]{\left\{#1\right\}} % curly parenthesis
\renewcommand{\v}{\bm} % bold vectors
\newcommand{\uv}[1]{\hat{\v{#1}}} % unit vectors
\renewcommand{\c}{\cdot} % inner product
\newcommand{\bk}{\Braket} % shorthand for braket notation

% shorthand for frequently used commands
\renewcommand{\u}{\uparrow}
\renewcommand{\d}{\downarrow}
\newcommand{\SWAP}{\t{SWAP}}
\newcommand{\cNOT}{\t{cNOT}}
\newcommand{\NV}{\t{NV}}
\newcommand{\AC}{\t{AC}}
\newcommand{\ST}{\t{ST}}
\renewcommand{\S}{\t{S}}
\newcommand{\T}{\t{T}}
\newcommand{\R}{\mathcal R}

%%% figures
\usepackage{graphicx,grffile,float} % floats, etc.
\usepackage{multirow} % multirow entries in tables
\usepackage{footnote} % footnotes in floating objects
\usepackage[font=small,labelfont=bf]{caption} % caption text options

%%% markup
\usepackage{color,soul} % text color and other editing options
% \ul{underline}, \st{strikethrough}, and \hl{highlight}
\newcommand{\fixme}[1]{{\bf \color{red} fixme: #1}}

\begin{document}

\title{Realization of a Decoherence-Free Subspace with
  Nitrogen-Vacancy Centers and $^{13}$C Nuclear Spins}

\author{Michael A. Perlin, Zhen-Yu Wang, Jorge Casanova, and Martin B. Plenio}

\affiliation{Institut f\"ur Theoretische Physik, Albert-Einstein-Allee
  11, Universit\"at Ulm, D-89069 Ulm, Germany}

\begin{abstract}
  Here is the abstract.
\end{abstract}

\maketitle

\section{Introduction}

\fixme{Write about NV centers, C-13 nuclei, introduction to larmor
  pairs and DFS, quantum memory/networks, etc.}

Nitrogen-vacancy (NV) centers in diamond have been recognized in
recent years as a promising platform for quantum sensing and quantum
information technologies\cite{mamin2013nanoscale, steinert2010high,
  wang2016positioning, chou2015optimal, childress2006fault,
  yao2012scalable}. The electronic ground and optically excited states
of NV$^-$ split into spin triplets which can be polarized, detected,
and coherently manipulated with high fidelity with fields and
microwave radiation \cite{dobrovitski2013quantum}. Hyperfine coupling
between the NV center electron and $^{13}$C nuclei in bulk diamond
allows for the coherent control and measurement of individual nuclear
spins, and the realization of universal set of high-fidelity quantum
gates \cite{dobrovitski2013quantum, casanova2016noise}.

One promising application of the NV center is as a memory register for
storing quantum information. While typical control operations on the
NV electron spin happen on the scale of nanoseconds, its
room-temperature coherence time is on the scale of microseconds
\cite{dobrovitski2013quantum}. Dynamic decoupling schemes and the use
of isotopically pure diamond samples can increase the NV electron spin
coherence time by a few orders of magnitude
\cite{ryan2010robust}. Storing quantum information in the NV electron
spin itself, however, reserves a valuable resource, as the center
could otherwise be used to address nuclear spins and other quantum
processing resources. The weaker environmental coupling of $^{13}$C
nuclei present in diamond yields much longer coherence times than that
of the NV electron spin, and by entangling multiple nuclear spins,
e.g. into the singlet-triplet states
$\p{\ket{\u\d}\pm\ket{\d\u}}/\sqrt2$, it is possible to generate
logical qubits which are protected from various sources of system and
environmental dephasing noise.

In this letter, we present a means to address individual $^{13}$C
nuclear spins which are normally indistinguishable due to a symmetry
in their coupling to the NV center. We use this addressing scheme to
initialize qubits in a decoherence-free subspace (DFS) of nuclear
spins, and propose a means for realizing a coherent SWAP operation
between the NV center electron spin state and the DFS qubit.

\section{Identifying symmetrically coupled $^{13}$C nuclear spins}

Restricting the NV electron spin states to the subspace spanned by
$\set{\ket0,\ket{m_s}}$ ($m_s=\pm1$), the AXY-$n$ pulse sequence
developed in by Casanova et al. in \cite{casanova2015robust} and
control scheme developed by Wang et al. in \cite{wang2016positioning}
allows addressing individual $^{13}$C nuclear spins via Hamiltonians
of the form
\begin{align}
  H_\t{int} = h_j^\t{int}\sigma_\NV^z\v I_j\c\uv n_j,
  \label{H_int}
\end{align}
for a tunable coupling strength $h_\t{int}$ and direction $\uv n_j$,
where $\sigma_\NV^z\equiv\op{m_s}-\op{0}$ and $\v I_j$ is the spin
operator for nuclear spin $j$. The protocol for realizing this
effective Hamiltonian, however, is conditional on a sufficiently
unique effective larmor frequency $\omega_j$, i.e. that for all
$n\ne j$
\begin{align}
  \abs{\omega_n-\omega_j} \gg \abs{f_{k_{DD}}\v A_j^\perp},
\end{align}
where $f_{k_{DD}}$ is the fourier component of the AXY-$n$ protocol on
resonance with $\omega_j$ and $\v A_j^\perp$ ($\v A_j^\parallel$) is
the component of the hyperfine vector $\v A_j$ perpendicular
(parallel) to the effective larmor precession axis $\uv\omega_j$.

It is possible, however, for two nuclear spins (indexed by 1 and 2) to
have identical larmor frequencies $\omega_1=\omega_2\equiv\omega$,
which occurs when both nuclei lie along the edges of a cylinder
centered on the NV center and oriented along the natural NV axis. We
call such nuclei {\it larmor pairs}, and their geometry results in
$A_1^\perp=A_2^\perp\equiv A^\perp$ and
$A_1^\parallel=A_2^\parallel\equiv A^\parallel$ (see Supplementary
Information) \fixme{prove in the SI}.

\fixme{Reference \cite{wang2016positioning} for identifying and
  location nuclei with unique larmor frequencies}

\fixme{Provide derivation for NV coherence when on resonance with the
  larmor frequency of a larmor pair}

\fixme{Provide derivation for NV coherence when on resonance with a
  larmor frequency when applying a control field for both the one- and
  two-nucleus case}

\fixme{Walk through protocol to identify and locate larmor pairs}

\section{Individually addressing nuclei in larmor pairs}

When trying to address one spin in a larmor pair, the protocols for
(\ref{H_int}) will simultaneously couple $\sigma_\NV^z$ to both
$\v I_1\c\uv n_1$ and $\v I_2\c\uv n_2$ with identical coupling
strength ($h_\t{int}$). So long as $\uv n_1$ and $\uv n_2$
(equivalently, $\v A_1^\perp$ and $\v A_2^\perp$) are not parallel,
however, it is possible to address only one of these spins at a time
by applying additional control fields. In particular, the application
of a magnetic decoupling field
$\v B_\t{dec}^{\p\omega}\p{t}=\v B_\t{dec}\cos\p{\omega
  t-\phi_\t{dec}}$ during the AXY-$n$ addressing protocol suppresses
components of $\uv n_j$ which are perpendicular to the vector
$\v B_\t{dec}$ projected onto the plane perpendicular to $\v\omega_j$
and rotated about the axis $\uv\omega_j$ by the angle
$\phi_\t{dec}$. One can therefore choose a direction $\v B_\t{dec}$
which completely suppresses NV electron spin coupling to spin 1 (2),
leaving only coupling to spin 2 (1).

\fixme{Give more information or details about this protocol? Reference
  SI}

\section{Qubit initialization in a decoherence-free subspace}

\fixme{Implement PRX\cite{reiserer2016robust} methods for initializing
  a qubit in $\ket{\u\d}\pm\ket{\d\u}$}

The state space of larmor pairs contains a subspace free of
decoherence via spin coupling to the NV center electron, i.e. a
decoherence-free subspace (DFS). This DFS is spanned by the
singlet-triplet states $\p{\ket{\u\d}\pm\ket{\d\u}}/\sqrt2$, or
equivalently by $\ket{\u\d}$ and $\ket{\d\u}$. \fixme{prove that this
  subspace is decoherence-free in the SI} The protocols for addressing
an individual nucleus in a larmor pair can be used to initialize a
qubit in this DFS. In particular, one can use the gate sequences in
\cite{reiserer2016robust} to deterministically initialize the larmor
pair into $\ket{\d\d}$, then probabilistically convert this state into
$\p{\ket{\u\d}+e^{i\phi}\ket{\d\u}}/\sqrt2$ for some phase $\phi$
determined by the choice of bases for the nuclear spins. While the
sequences to initialize into $\ket{\d\d}$ require individually
addressing each nuclear spin, the conversion
$\ket{\d\d}\to\p{\ket{\u\d}+e^{i\phi}\ket{\d\u}}/\sqrt2$ can be
performed with simultaneous coupling to both larmor pairs,
i.e. without the magnetic decoupling field.

\section{Coherent SWAP operation with DFS qubit}



\bibliography{\jobname}


\pagebreak
\clearpage
\widetext
\makeatletter
\begin{center}
  \large \bf Supplementary Information for ``\@title''
\end{center}
\setcounter{equation}{0}
\setcounter{figure}{0}
\setcounter{table}{0}
\setcounter{page}{1}
\renewcommand{\theequation}{S\arabic{equation}}
\renewcommand{\thefigure}{S\arabic{figure}}
\renewcommand{\bibnumfmt}[1]{[S#1]}
\renewcommand{\citenumfont}[1]{S#1}

\tableofcontents{}

\section{Theory}

\subsection{Basic system model}

The Hamiltonian of the NV center system in the presence of a strong,
static magnetic field $\v B$ can be decomposed as
\begin{align}
  H_\t{sys} = H_\NV^\t{GS} + H_\t{hf} + H_\t{nZ}  + H_\t{nn},
\end{align}
where
\begin{align}
  H_\NV^\t{GS} = D\p{\v S\c\uv z}^2 - \gamma_e\v B\c\v S
\end{align}
is the ground-state NV electron spin Hamiltonian with zero-field
splitting $D$, spin-1 operator $\v S$, natural NV axis $\uv z$, and
electron gyromagnetic ratio $\gamma_e$;
\begin{align}
  H_\t{hf} = \sum_j\f{\gamma_e\gamma_j}{4\pi r_j^3}
  \p{\v S\c\v I_j-3\sp{\v S\c\uv r_j}\sp{\v I_j\c\uv r_j}}
\end{align}
is the hyperfine coupling between the NV electron spin (located at the
origin) and nuclei indexed by $j$ located at $\v r_j$;
\begin{align}
  H_\t{nZ} = -\sum_j\gamma_j\v B\c\v I_j
\end{align}
is the Zeeman Hamiltonian for nuclei indexed by $j$ with gyromagnetic
ratio $\gamma_j$ and spin operators $\v I_j$; and
\begin{align}
  H_\t{nn} = \sum_{j<k}\f{\gamma_j\gamma_k}{4\pi r_{jk}^3}
  \p{\v I_j\c\v I_k-3\sp{\v I_j\c\uv r_{jk}}\sp{\v I_k\c\uv r_{jk}}}
\end{align}
is the internuclear coupling with $\v r_{jk}\equiv\v r_j-\v r_k$. We
will hereafter neglect internuclear coupling in all analytical work,
though we will keep $H_\t{nn}$ in any numerical results.

Changing into the rotating frame of $H_\NV^\t{GS}$ affects only the
hyperfine Hamiltonian, which becomes
\begin{align}
  \tilde H_\t{hf} = S_z\sum_j \v A_j\c\v I_j,
\end{align}
where we define the hyperfine field
\begin{align}
  \v A_j = \f{\gamma_e\gamma_j}{4\pi r_j^3}\p{\uv z - 3\sp{\uv
  r_j\c\uv z}\uv r_j}.
\end{align}
The system Hamiltonian is thus
\begin{align}
  \tilde H_\t{sys} = S_z\sum_j \v A_j\c\v I_j
  -\sum_j\gamma_j\v B\c\v I_j.
\end{align}

When restricted to the subspace of $\set{\ket0,\ket{m_s}}$ with with
$m_s\in\set{1,-1}$ for the NV electron spin,
\begin{align}
  S_z = \f{m_s}2\p{\sigma_\NV^z+\mathbf1},
\end{align}
where $\sigma_\NV^z\equiv\op{m_s}-\op{0}$ and
$\mathbf1\equiv\op{m_s}+\op{0}$. Defining the interaction Hamiltonian
\begin{align}
  H_\t{int} = \f12m_s\sigma_\NV^z\sum_j\v A_j\c\v I_j
\end{align}
the effective larmor frequency
\begin{align}
  \v\omega_j = \gamma_j\v B - \f{m_s}2\v A_j,
\end{align}
and the effective Zeeman Hamiltonian
\begin{align}
  H_\t{nZ}^\t{eff} = -\sum_j\p{\gamma_j\v B - \f{m_s}2\v A_j}\c\v I_j
  = -\sum_j\v\omega_j\c\v I_j,
\end{align}
the system Hamiltonian can be written as
\begin{align}
  \tilde H_\t{sys} = H_\t{int} + H_\t{nZ}^\t{eff}.
  \label{S_H_sys_int_nZ}
\end{align}

Moving into the rotating frame of $H_\t{nZ}^\t{eff}$, the entire
system is described by the Hamiltonian
\begin{align}
  \tilde H_\t{int} = \f12m_s\sigma_\NV^z
  \sum_j\tilde{\v A_j}\p{t}\c\v I_j,
  \label{S_H_int_no_DD}
\end{align}
where $\tilde{\v A_j}\p{t}$ is the vector $\v A_j$ rotated about
$\uv\omega_j$ by an angle $\omega_jt$. Written explicitly,
\begin{align}
  \tilde{\v A_j}\p{t} = \v A_j^\parallel + \v A_j^\perp\cos\p{\omega_jt}
  + \v A_j^{\perp\perp}\sin\p{\omega_jt},
  \label{S_A_rot}
\end{align}
where
\begin{align}
  \v A_j^\parallel\equiv\p{\v A_j\c\uv\omega_j}\uv\omega_j
\end{align}
is the projection of $\v A_j$ along $\v \omega_j$,
\begin{align}
  \v A_j^\perp\equiv\v A_j-\v A_j^\parallel
\end{align}
is the projection of $\v A_j$ onto the plane orthogonal to
$\v\omega_j$, and
\begin{align}
  \v A_j^{\perp\perp}\equiv\uv\omega_j\times\v A_j^\perp
\end{align}
is the vector $\v A_j^\perp$ rotated about $\uv\omega_j$ by $\pi/2$.

\subsection{Dynamic decoupling and nuclear spin addressing}

Dynamic decoupling schemes effectively prepend a modulation function
\begin{align}
  F\p{t}=\sum_kf_k\cos\p{k\omega_{DD}t-\phi_{DD}}
\end{align}
with fundamental frequency $\omega_{DD}$, phase $\phi_{DD}$, and range
$\set{1,-1}$ to $\sigma_\NV^z$, so that the system Hamiltonian reads
\begin{align}
  \tilde H_\t{int} =
  \f12m_sF\p{t}\sigma_\NV^z\sum_j\tilde{\v A_j}\p{t}\c\v I_j.
\end{align}
At a resonance $k_{DD}\omega_{DD}\equiv\omega_r\gg A_j$, the secular
approximation allows one to neglect $\v A_j^\parallel$ and express
\begin{align}
  \tilde H_\t{int} = \f14 f_{k_{DD}} m_s\sigma_\NV^z
  \sum_{j:\omega_j=\omega_r} A_j^\perp\v I_j\c\uv n_j,
  \label{S_H_int_resonance}
\end{align}
where $\uv n_j$ is a unit vector in the direction of $\v A_j^\perp$
rotated about $\uv\omega_j$ by $\phi_{DD}$, i.e.
\begin{align}
  \uv n_j = \uv A_j^\perp\cos\phi_{DD} +
  \uv A_j^{\perp\perp}\sin\phi_{DD}.
\end{align}
One can concisely write (\ref{S_H_int_resonance}) with
\begin{align}
  h_j^\t{int}\equiv f_{k_{DD}}m_sA_j^\perp/4
\end{align}
as
\begin{align}
  \tilde H_\t{int} = \sum_{j:\omega_j=\omega_r}
  h_j^\t{int}\sigma_\NV^z\v I_j\c\uv n_j.
  \label{S_H_int}
\end{align}

When the harmonics of $F\p{t}$ are off resonance with all $\omega_j$
and $\omega_{DD}\gg A_j$ for all $j$, the secular approximation allows
one to neglect $\tilde H_\t{int}$ entirely. In this case, one can
apply an auxiliary magnetic field
$\v B_\t{ctl}^{\p\omega}\p{t}=\v
B_\t{ctl}\cos\p{\omega_\t{ctl}t-\phi_\t{ctl}}$ and shift into the
rotating frame of $H_\t{nZ}^\t{eff}$, much like the frame shift
between (\ref{S_H_sys_int_nZ}) and (\ref{S_H_int_no_DD}), to realize
the Hamiltonian
\begin{align}
  \tilde H_\t{sys}^\t{ctl} =
  -\sum_j\gamma_j\tilde{\v B}_{\t{ctl},j}\p{t}
  \cos\p{\omega_\t{ctl}t-\phi_\t{ctl}}\c\v I_j,
\end{align}
with $\tilde{\v B}_{\t{ctl},j}\p{t}$ defined using $\v B_\t{ctl}$ and
$\v\omega_j$ the same way that $\tilde{\v A_j}\p{t}$ was defined using
$\v A_j$ and $\v\omega_j$ in (\ref{S_A_rot}), i.e. as $\v B_\t{ctl}$
rotated about $\uv\omega_j$ by an angle $\omega_jt$:
\begin{align}
  \tilde{\v B}_{\t{ctl},j}\p{t} = \v B_{\t{ctl},j}^\parallel
  + \v B_{\t{ctl},j}^\perp\cos\p{\omega_jt}
  + \v B_{\t{ctl},j}^{\perp\perp}\sin\p{\omega_jt},
\end{align}
where
\begin{align}
  \v B_{\t{ctl},j}^\parallel
  &\equiv \p{\v B_\t{ctl} \c\uv\omega_j}\uv\omega_j \\
  \v B_{\t{ctl},j}^\perp
  &\equiv \v B_\t{ctl} - \v B_{\t{ctl},j}^\parallel \\
  \v B_{\t{ctl},j}^{\perp\perp}
  &\equiv \uv\omega_j\times\v B_\t{ctl}^\perp.
\end{align}

When $\omega_\t{ctl}\gg\gamma_jB_\t{ctl}$, the secular approximation
allows one to neglect $\v B_{\t{ctl},j}^\parallel$ and express
\begin{align}
  \tilde H_\t{ctl} = -\f12\sum_{j:\omega_j=\omega_\t{ctl}}
  \gamma_jB_{\t{ctl},j}^\perp \v I_j\c\uv m_j,
  \label{S_H_ctl_resonance}
\end{align}
where $\uv m_j$ is the direction of $\v B_{\t{ctl},j}^\perp$ rotated
about $\uv\omega_j$ by $\phi_\t{ctl}$, i.e.
\begin{align}
  \uv m_j = \uv B_{\t{ctl},j}^\perp\cos\phi_\t{ctl} +
  \uv B_{\t{ctl},j}^{\perp\perp}\sin\phi_\t{ctl}.
\end{align}
One can concisely write (\ref{S_H_ctl_resonance}) with
\begin{align}
  h_j^\t{ctl}\equiv-\gamma_jB_{\t{ctl},j}^\perp/2
\end{align}
as
\begin{align}
  \tilde H_\t{ctl} = \sum_{j:\omega_j=\omega_\t{ctl}}
  h_j^\t{ctl}\v I_j\c\uv m_j.
  \label{S_H_ctl}
\end{align}

In both (\ref{S_H_int}) and (\ref{S_H_ctl}), one can choose
$\phi_{DD}$ and $\phi_\t{ctl}$ such that $\uv n_j\p\phi$ and
$\uv m_j\p{\phi_\t{ctl}}$ lie anywhere in the plane perpendicular to
$\uv\omega_j$. One can also tune $h_j^\t{int}$ and $h_j^\t{ctl}$
respectively via free parameters in the AXY-$n$ sequence and
$\v B_\t{ctl}$. These controls, together with rotations of the NV
electron spin, allow one to realize a universal set of quantum
gates. An exception to this statement is when two or more nuclei have
the same effective larmor frequency $\omega_j$, in which case they are
addressed simultaneously in (\ref{S_H_int}) and (\ref{S_H_ctl}).

\subsection{Addressing larmor pairs}

We call two spins indexed by $j$ and $k$ elements of a \emph{larmor
  set} if their effective larmor frequencies are equal,
i.e. $\omega_j=\omega_k$, and refer to a larmor set with exactly two
spins as a \emph{larmor pair}. The problem with individually
addressing elements of a larmor set is manifest in (\ref{S_H_int}) and
(\ref{S_H_ctl}), wherein the previously developed procedure for
nuclear spin addressing does not single out any nuclei. Nonetheless,
under certain conditions it is possible to individually address
elements of a larmor set by applying additional control fields.

When employing a dynamic decoupling protocol on resonance with an
effective larmor frequency $\omega_r$, one can apply an additional
magnetic decoupling field
$\v B_\t{dec}^{\p\omega}\p{t}=\v B_\t{dec}\cos\p{\omega_r
  t-\phi_\t{dec}}$ and realize, in the frame of $H_\NV^\t{GS}$, the
system Hamiltonian
\begin{align}
  \tilde H_\t{sys}^\t{dec} = H_\t{int} + H_\t{nZ}^\t{eff} + H_\t{dec},
\end{align}
where
\begin{align}
  H_\t{dec} = -\sum_j\gamma_j\v B_\t{dec}\c\v I_j
  \cos\p{\omega_rt-\phi_\t{dec}}.
\end{align}

In the frame of $H_\t{nZ}^\t{eff}=-\sum_j\v\omega_j\c\v I_j$, the
decoupling Hamiltonian becomes
\begin{align}
  \tilde H_\t{dec} = -\sum_j\gamma_j\tilde{\v B}_{\t{dec},j}
  \c\v I_j \cos\p{\omega_rt-\phi_\t{dec}}
  \label{S_H_dec_eff_full}
\end{align}
with
\begin{align}
  \tilde{\v B}_{\t{dec},j} = \v B_{\t{dec},j}^\parallel
  + \v B_{\t{dec},j}^\perp\cos\p{\omega_j t}
  + \v B_{\t{dec},j}^{\perp\perp}\sin\p{\omega_j t}
\end{align}
and
\begin{align}
  \v B_{\t{dec},j}^\parallel
  &\equiv \p{\v B_\t{dec} \c\uv\omega_j}\uv\omega_j, \\
  \v B_{\t{dec},j}^\perp
  &\equiv \v B_\t{dec} - \v B_{\t{dec},j}^\parallel, \\
  \v B_{\t{dec},j}^{\perp\perp}
  &\equiv \uv\omega_j\times\v B_\t{dec}^\perp.
\end{align}
If $\omega_r,\omega_j\gg \gamma_jB_\t{dec}$, then we can neglect
$\v B_{\t{dec},j}^\parallel$ in (\ref{S_H_dec_eff_full}) by the
secular approximation, and by the rotating approximation express
\begin{align}
  \tilde H_\t{dec} = -\sum_{j:\omega_j=\omega_r}
  \f12\gamma_j\p{\v B_{\t{dec},j}^\perp\cos\phi_\t{dec} +
  \v B_{\t{dec},j}^{\perp\perp}\sin\phi_\t{dec}}\c\v I_j.
\end{align}
Defining
\begin{align}
  \v\nu_j = \f12\gamma_j\p{\v B_{\t{dec},j}^\perp\cos\phi_\t{dec}
  + \v B_{\t{dec},j}^{\perp\perp}\sin\phi_\t{dec}},
\end{align}
we can express
\begin{align}
  \tilde H_\t{dec} = -\sum_{j:\omega_j=\omega_r}\v\nu_j\c\v I_j.
\end{align}
The system Hamiltonian in the frame of $H_\t{nZ}^\t{eff}$ thus takes
the form
\begin{align}
  \tilde H_\t{sys}^\t{dec} = \tilde H_\t{int} + \tilde H_\t{dec}
  = \sum_{j:\omega_j=\omega_r}
  \p{h_j^\t{int}\sigma_\NV^z\v I_j\c\uv n_j - \v\nu_j\c\v I_j}.
\end{align}

In the frame of $\tilde H_\t{dec}$, the entire system is governed by
the Hamiltonian
\begin{align}
  \tilde H_\t{int}^\t{dec} = \sum_{j:\omega_j=\omega_r}
  h_j^\t{int}\sigma_\NV^z\v I_j\c\tilde{\v n}_j\p{t},
  \label{S_H_int_dec_full}
\end{align}
where $\tilde{\v n}_j\p{t}$ is the unit vector $\uv n_j$ rotated about
$\uv\nu_j$ by $\nu_jt$, i.e.
\begin{align}
  \tilde{\v n}_j\p{t} = \v n_j^\parallel +
  \v n_j^\perp\cos\p{\nu_jt} + \v n_j^{\perp\perp}\sin\p{\nu_jt},
\end{align}
with
\begin{align}
  \v n_j^\parallel &\equiv \p{\uv n_j\c\uv\nu_j}\uv\nu_j, \\
  \v n_j^\perp &\equiv \uv n_j - \uv n_j^\parallel, \\
  \v n_j^{\perp\perp} &\equiv \uv\nu_j\times\uv n_j.
\end{align}
If $\omega_j\gg\nu_j\gg h_j^\t{int}$, the secular approximation allows
us to neglect $\v n_j^\perp$ and $\v n_j^{\perp\perp}$ in
(\ref{S_H_int_dec_full}), leaving
\begin{align}
  \tilde H_\t{int}^\t{dec}
  = \sum_{j:\omega_j=\omega_r}
  \tilde h_j^\t{int}\sigma_\NV^z\v I_j\c\uv\nu_j,
  \label{S_H_int_dec}
\end{align}
where
\begin{align}
  \tilde h_j^\t{int} = h_j^\t{int}\uv n_j\c\uv\nu_j.
\end{align}

In words, the decoupling field $\v B_\t{dec}^{\p\omega}\p{t}$ modifies
$\tilde H_\t{int}$ in (\ref{S_H_int}) by suppressing the components of
$\uv n_j$ (i.e. $\uv A_j^\perp$ rotated about $\uv\omega_j$ by
$\phi_{DD}$) which are perpendicular to the respective $\uv\nu_j$
(i.e.  $\uv B_{\t{dec},j}^\perp$ rotated about $\uv\omega_j$ by
$\phi_\t{dec}$). The utility of this decoupling field lies in the fact
that when $\uv n_j$ and $\uv\nu_j$ are orthogonal, the interaction
with spin $j$ is suppressed as $\tilde h_j^\t{int}=0$. Considering a
single larmor pair, one can therefore apply a decoupling field to
eliminate one of the two terms in (\ref{S_H_int_dec}) and realize an
effective Hamiltonian which couples the NV electron spin to only one
nuclear spin. This scheme fails for larmor pairs with mutually
parallel $\v A_j^\perp$.

The tools developed in this paper do not allow for single-spin
operations on elements of a larmor pair. Such operations must
therefore be either forgone, or else realized via composite operations
making use of (\ref{S_H_ctl}) and (\ref{S_H_int_dec}). The same is
true of spin-spin coupling between the NV electron spin and a single
element of a larmor set with more than two nuclei.

\subsection{Coherent SWAP with a singlet-triplet qubit}

For the following analysis, we will consider a single addressable
larmor pair, i.e. a pair of nuclei indexed by 1 and 2 with effective
larmor frequencies $\omega$ and for which $\v A_j^\perp$ are not
mutually parallel. We will use the basis
$\set{\ket\u\equiv\ket 0, \ket\d\equiv\ket{m_s}}$ for the spin states
of the NV electron, and the standard basis
$\set{\ket{\u\u}, \ket{\u\d}, \ket{\d\u}, \ket{\d\d}}$ for the spin
states of the nuclei. We will refer to the logical qubit consisting of
the basis states $\set{\ket{\u\d},\ket{\d\u}}$ as the anti-correlated
(AC) qubit, and the logical qubit consisting of the basis states
$\set{\ket\S,\ket\T}$ as the singlet-triplet (ST) qubit, where
\begin{align}
  \ket\S &\equiv \f1{\sqrt2}\p{\ket{\u\d}-\ket{\d\u}}, \\
  \ket\T &\equiv \f1{\sqrt2}\p{\ket{\u\d}+\ket{\d\u}}.
\end{align}

A SWAP operation between two qubits indexed by 1 and 2 can be realized
via the composite operation
\begin{align}
  \SWAP^{1,2} = \cNOT^{2\to1}\cNOT^{1\to2}\cNOT^{2\to1},
\end{align}
where $\cNOT^{j\to k}$ is the controlled-NOT gate, which flips the
state of qubit $k$ conditionally on the state of qubit $j$. The
$\cNOT^{j\to k}$ gate is realized via
\begin{align}
  \cNOT^{j\to k} = \exp\p{-j\f\pi2I_j^z} \exp\p{-j\f\pi2I_k^x}
  \exp\p{j\pi I_j^z I_k^x}.
\end{align}

If we wish to perform a SWAP operation between the NV electron spin
and an ST qubit, it is easier to first realize the operation
\begin{align}
  \SWAP^{\NV,\AC}
  = {\cNOT^{\AC\to\NV}}^\dagger\cNOT^{\NV\to\AC}\cNOT^{\AC\to\NV},
  \label{S_SWAP_NVAC_sketch}
\end{align}
where
\begin{align}
  \cNOT^{\AC\to\NV} = \cNOT^{1\to\NV}
\end{align}
and
\begin{align}
  \cNOT^{\NV\to\AC} = \cNOT^{\NV\to2}\cNOT^{\NV\to1}.
\end{align}
For later convenience, (\ref{S_SWAP_NVAC_sketch}) makes use of the
fact that ${\cNOT^{j\to k}}^\dagger=\cNOT^{j\to k}$. Therefore (with
$\v I_\NV=\v\sigma_\NV/2$),
\begin{align}
  \SWAP^{\NV,\AC}
  =& \exp\p{-j\f\pi2 I_\NV^x} \exp\p{j\f\pi2 I_1^{z'}}
     \exp\p{-j\f\pi2 \sigma_\NV^x I_1^{z'}} \times \tag*{} \\
   & \exp\p{-j\f\pi2 I_\NV^z} \exp\p{-j\f\pi2 I_1^{x'}}
     \exp\p{j\f\pi2 \sigma_\NV^z I_1^{x'}} \times \tag*{} \\
   & \exp\p{-j\f\pi2 I_\NV^z} \exp\p{-j\f\pi2 I_2^{x''}}
     \exp\p{j\f\pi2 \sigma_\NV^z I_2^{x''}} \times \tag*{} \\
   & \exp\p{-j\f\pi2 I_\NV^x} \exp\p{-j\f\pi2 I_1^{z'}}
     \exp\p{j\f\pi2 \sigma_\NV^x I_1^{z'}},
  \label{S_SWAP_NVAC_full}
\end{align}
where we make explicit the fact that the the spin of nuclei 1 and 2
may be given in their own respective bases
$\set{\uv x',\uv y',\uv z'}$ and $\set{\uv x'',\uv y'',\uv z''}$. We
can evaluate the first and last rotations of spin 1 in
(\ref{S_SWAP_NVAC_full}) and combine factors to get
\begin{align}
  \SWAP^{\NV,\AC}
  =& \exp\p{j\f\pi2 I_\NV^x} \exp\p{-j\f\pi2 \sigma_\NV^x I_1^{z'}}
     \times \tag*{} \\
   & \exp\p{-j\pi I_\NV^z} \exp\p{j\f\pi2\sp{I_1^{y'}-I_2^{x''}}}
     \times \tag*{} \\
   & \exp\p{-j\f\pi2 \sigma_\NV^z I_1^{y'}}
     \exp\p{j\f\pi2 \sigma_\NV^z I_2^{x''}} \times \tag*{} \\
   & \exp\p{j\f\pi2 \sigma_\NV^x I_1^{z'}} \exp\p{-j\f\pi2 I_\NV^x}.
\end{align}
Letting
\begin{align}
  \uv x' = \uv y'' &= \uv A_1^\perp, \\
  \uv y' = -\uv x'' &= \uv A_1^{\perp\perp}, \\
  \uv z' = \uv z'' &= \uv\omega_1\approx\uv z, \label{S_w=z}
\end{align}
where the approximation in (\ref{S_w=z}) holds for
$\gamma_1B\gg A_1/2$,
\begin{align}
  \SWAP^{\NV,\AC}
  =& \exp\p{j\f\pi2 I_\NV^x}
     \exp\p{-j\f\pi2 \sigma_\NV^x I_1^z} \times \tag*{} \\
   & \exp\p{-j\pi I_\NV^z}
     \exp\p{j\f\pi2\sum_jI_j^{y'}} \times \tag*{} \\
   & \exp\p{-j\f\pi2 \sigma_\NV^z I_1^{y'}}
     \exp\p{-j\f\pi2 \sigma_\NV^z I_2^{y'}} \times \tag*{} \\
   & \exp\p{j\f\pi2 \sigma_\NV^x I_1^z} \exp\p{-j\f\pi2 I_\NV^x}.
  \label{S_SWAP_NVAC_sxsz}
\end{align}
As we cannot directly realize the coupling $\sigma_x^\NV I_z^1$,
we must perform rotations as
\begin{align}
  \exp\p{-j\f\pi2 \sigma_\NV^x I_1^z}
  =& \exp\p{-j\f\pi2 I_\NV^y} \exp\p{j\f\pi2 I_1^{y'}}
     \exp\p{-j\f\pi2 \sigma_\NV^z I_1^{x'}}
     \exp\p{-j\f\pi2 I_1^{y'}} \exp\p{j\f\pi2 I_\NV^y}.
\end{align}
Substituting this identity into (\ref{S_SWAP_NVAC_sxsz}) yields
\begin{align}
  \SWAP^{\NV,\AC}
  =& \exp\p{j\f\pi2 I_\NV^x} \exp\p{-j\f\pi2 I_\NV^y} \times \tag*{} \\
   & \exp\p{j\f\pi2\sum_jI_j^{y'}}
     \exp\p{-j\f\pi2 \sigma_\NV^z I_1^{x'}}
     \exp\p{j\f\pi2 I_\NV^y} \times \tag*{} \\
   & \exp\p{-j\pi I_\NV^z}
     \exp\p{j\f\pi2\sum_jI_j^{y'}} \times \tag*{} \\
   & \exp\p{-j\f\pi2 \sigma_\NV^z I_1^{y'}}
     \exp\p{-j\f\pi2 \sigma_\NV^z I_2^{y'}} \times \tag*{} \\
   & \exp\p{-j\f\pi2 I_\NV^y}
     \exp\p{j\f\pi2 \sigma_\NV^z I_1^{x'}}
     \exp\p{-j\f\pi2\sum_jI_j^{y'}} \times \tag*{} \\
   & \exp\p{j\f\pi2 I_\NV^y} \exp\p{-j\f\pi2 I_\NV^x},
  \label{S_SWAP_NVAC}
\end{align}
where we have canceled out two rotations of the form
$\exp\p{\pm j\pi/2~I_1^{y'}}$ and inserted two rotations of the form
$\exp\p{\pm j\pi/2~I_2^{y'}}$ with no net effect.

The $\SWAP^{\NV,\ST}$ can be realized using the $\SWAP^{\NV,\AC}$ gate
via
\begin{align}
  \SWAP^{\NV,\ST} = R_\NV^\dagger \SWAP^{\NV,\AC} R_\NV,
\end{align}
where the rotation $R_\NV$ takes $\ket\u\to\p{\ket\u-\ket\d}/\sqrt2$
and $\ket\d\to\p{\ket\u+\ket\d}/\sqrt2$. Written explicitly,
\begin{align}
  R_\NV = \exp\p{j\f\pi2 I_\NV^y}.
\end{align}
We therefore have that
\begin{align}
  \SWAP^{\NV,\ST}
  =& \exp\p{j\pi\v I_\NV\c\uv a} \times \tag*{} \\
   & \exp\p{j\f\pi2\sum_jI_j^{y'}}
     \exp\p{-j\f\pi2 \sigma_\NV^z I_1^{x'}}
     \exp\p{j\f\pi2 I_\NV^y} \times \tag*{} \\
   & \exp\p{-j\pi I_\NV^z}
     \exp\p{j\f\pi2\sum_jI_j^{y'}} \times \tag*{} \\
   & \exp\p{-j\f\pi2 \sigma_\NV^z I_1^{y'}}
     \exp\p{-j\f\pi2 \sigma_\NV^z I_2^{y'}} \times \tag*{} \\
   & \exp\p{-j\f\pi2 I_\NV^y}
     \exp\p{j\f\pi2 \sigma_\NV^z I_1^{x'}}
     \exp\p{-j\f\pi2\sum_jI_j^{y'}} \times \tag*{} \\
   & \exp\p{-j\pi\v I_\NV\c\uv a},
  \label{S_SWAP_NVST}
\end{align}
where $\v a = \uv x - \uv y$ and we have used the fact that
\begin{align}
  \exp\p{j\f\pi2 I^y} \exp\p{-j\f\pi2 I^x}
  \exp\p{j\f\pi2 I^y}
  = \exp\p{-j\pi\v I\c\uv a}.
\end{align}

In the subspace of interest, we have thus constructed the
non-entangling SWAP operation
\begin{align}
  \SWAP^{\NV,\ST} = \op{\u\S}{\S\u} + \op{\u\T}{\d\S}
  + \op{\d\S}{\u\T} + \op{\d\T}{\d\T}.
\end{align}
If $\gamma_1=\gamma_2$, i.e. if nuclei 1 and 2 are of the same
species, all factors in (\ref{S_SWAP_NVST}) can be realized directly
via the protocols for (\ref{S_H_ctl}), (\ref{S_H_int_dec}), and
rotations of the NV electron spin.

Realizing $\SWAP^{\NV,\AC}$ by switching the cNOT operations in
(\ref{S_SWAP_NVAC_sketch}) is seemingly impossible, as doing so
requires rotating one spin in the larmor pair without rotating the
other. It is possible to realize $\SWAP^{\NV,\ST}$ via cNOT operations
between the NV and ST qubits directly, but this method requires
performing many more individual spin addressing operations, and
therefore has a substantially lower fidelity.

\fixme{when we wish to read out the ST qubit, it is possible to do so
  without performing the full coherent SWAP operation so long as we no
  longer need the larmor pair as a resource for the time being. Talk
  about this.}

\subsection{Robustness against noise}

Talk about the robustness of the ST qubit against dephasing via the NV
center, as well as its robustness against slowly varying (frequencies
$\nu\ll\gamma_jB$) external magnetic noise.

\section{Simulations and Results}

\subsection{Abundance of larmor pairs}

For any given hyperfine cutoff $A_\t{min}$ and an isotopic abundance
$n$ of $^{13}$C, we can compute exactly the probability
$P\p{A_\t{min},n}$ of finding an addressable larmor pair for which
$A_j\ge A_\t{min}$. The simulations developed for this paper can also
be used to test the computation of $P\p{A_\t{min},n}$ via the Monte
Carlo method of simply generating many NV systems and counting the
proportion of them which contain an addressable larmor pair.

\subsection{SWAP fidelities via AXY-$n$}

Discuss simulations and any relevant nuances therein. Present
histograms of $\SWAP^{\NV,\ST}$ fidelities and net operation times.

Note: the same simulations can also be used to generate similar
histograms (as well as other info) about other operations,
e.g. single-spin $\SWAP$ or $i\SWAP$ operations. Furthermore, the
simulations can, without any modification, reproduce results such as
Figure 2 in \cite{wang2016positioning} (``Positioning Nuclear Spins in
Interacting Clusters for Quantum Technologies and
Bio-imaging''). Minor additions to the codes might make them useful
for comparing the efficacy of different spin-addressing protocols. Is
there any interest in doing any of these things (for this paper or
elsewhere)?

\section{Conclusions}

\section{Acknowledgments}

\bibliography{\jobname}

\end{document}


notes:

prove DFS

how to locate spins?
- identify larmor frequencies for C-13 nuclei
- identify a frequency with two spins
- locate two spins

perform elementary gate on one spin
implement PRX gates